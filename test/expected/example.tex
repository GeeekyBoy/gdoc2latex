\title{%
  \textbf{gdoc2latex}
  \linebreak \linebreak
  \large{Converts Google Docs files to LaTeX}
}\documentclass[12pt]{article}
\usepackage[a4paper,left=2.54cm,right=2.54cm,top=2.54cm,bottom=2.54cm]{geometry}

\usepackage[T1]{fontenc}
\usepackage[utf8]{inputenc}
\usepackage{lmodern}

\setlength\parindent{0em}
\usepackage{parskip}
\setcounter{tocdepth}{1}

\usepackage{hyperref}

\author{
    Your Name \\
    Department of Something \\
    University of Somewhere
}
\date{}

\begin{document}


\maketitle

\section{Usage}

Here’s some content, originally written in Google Docs. This was then downloaded with File > Download > Web page (.html)

\section{Supported features}

Things like \textbf{bold} and \textit{italics} among other things are supported.

Other features include:

\begin{itemize}
  \item Lists
  \item \underline{Underline}
  \item References (use BibTeX in a Google Docs footnote)\cite{ref1}
  \item Footnotes\footnote{Like this! One limitation though, they can’t start with an @ symbol}
  \item Both \textsuperscript{superscript} and \textsubscript{subscript}
\end{itemize}

\bibliography{example}

\bibliographystyle{abbrvurl}

\end{document}